\subsection{Komande}

\textbf{Kontrola pristupa}\\
\textit{line console 0} - Lozinka za pristup preko konzole\\
\textit{line aux 0} - Lozinka za pristup preko pomo\'{c}nog\\
\textit{line vty 0 15} - Lozinka za pristup putem Telneta – VTY\\
\textit{password X} - Unos lozinke - Ide nakon svih komandi iznad\\
\textit{login} - Ide nakon svih komandi iznad\\
\textit{enable password X} - Lozinka za pristup privilegovanom režimu\\
\textit{enable secret Y} - Šifrovana lozinka za pristup privilegovanom režimu. Šifrovana lozinka mora biti drugačija od \textit{plain text} lozinke\\\\

\textbf{Osnovno mreženje}\\
\textit{interface X} - Bira interfejs\\
\textit{clock rate X} - Naglašava brzinu pisanja podataka na serijskoj konekciji; Podešava se na DCE ruteru; Najčeš\'{c}a brzina je: 4000000\\
\textit{ip address x.x.x.x y.y.y.y} - Posavlja ličnu IP adresu na interfejsu. X predstavlja IP adresu a Y mrežnu masku\\
\textit{no shutdown} - Naglašava da se interfejs ne gasi\\\\

\textbf{DHCP}\\
\textit{ip dhcp excluded-address x.x.x.x x.x.x.x} - Naglašava opseg IP adresa koje ne\'{c}e mo\'{c}i da se dodele uređajima kroz DHCP\\
\textit{ip dhcp pool X} - Bira DHCP "Pool"\ za podešavanje parametara\\
\textit{network x.x.x.x y.y.y.y} - Podešava koju mrežu \'{c}e DHCP koristiti. X predstavlja IP adresu a Y mrežnu masku\\
\textit{default-router x.x.x.x} - Podešava ulazni ruter u DHCP mrežu\\
\textit{ip route x.x.x.x y.y.y.y z.z.z.z} - Podešava rutiranje do privatne mreže. X - Privatna adresa, Y - Mrežna maska priv. adrese, Z - Adresa rutera preko kog se pristupa mreži\\\\

\textbf{NAT}\\
\textit{ip nat inside source static x.x.x.x y.y.y.y} - Mapira privatnu (x) ka javnoj (y) adresi\\
\textit{ip nat X} - Podešava interfejs rutera da je sa unutrašnje (inside) ili spoljašnje (outside) strane NAT\\

\textbf{PAT}\\
\textit{ip access list standard X} - Podešava PAT listu\\
\textit{permit x.x.x.x y.y.y.y} - Podešava koje mrežne adrese trebaju da se prevode. X je adresa a Y wildcard biti (255.255.255.255 - podmreža)\\
\textit{ip nat inside source list X int Y overload} - Mapira listu (X) privatnih adresa ka spoljašnjem interfejsu (Y) NAT\\